\documentclass{article} \usepackage{a4}
\usepackage{epsfig}
\usepackage{float}
\usepackage{chicago}
\bibliographystyle{chicago}
%\bibliographystyle{plain}

\renewcommand{\baselinestretch}{2}
%\setcounter{secnumdepth}{0}

% this is the preamble
\title{Promoterwise: flexible alignment of promoters and its application to Mammalian vs Telost comparisons} 
\author{L. Ettwiller, B. Paten and E. Birney}
\begin{document}



\maketitle
 
\section{Abstract}

..to be written..

\section{Introduction}

The alignment of DNA sequences is one of the earliest applications of
bioinformatics \shortcite{gotoh,smithwaterman,needleman} and remains a
core features of sequence analysis. The original alignment methods
have focused principally on the alignment of coding sequences, either
at the DNA level or at the protein level, mainly because the vast
majority of available sequences were involved with protein coding
genes. More recently there has been an interest in aligning non coding
DNA driven by the availability of large genomes with considerable non
coding extents, for example, 95\% of bases for mammalian species are
thought to be non coding. A considerable amount of the complexity of
large metozoan genomes is thought to be encoded in non-coding
DNA. This currently poorly understood additional code in the genome
should be revealed by the action of both negative and positive
selection on specific sequences, giving rise in each case to specific
patterns of alignments which deviate from an underlying supposed
neutral model of base pair evolution (for a review, see
\shortcite{UretaVidal}).

Given such a widespread interest in non coding alignments, a number of
groups have applied standard methods to the alignment of genome
sequences \shortcite{?}, developed new methods to align genome
sequences \shortcite{Jareborg}, or both \shortcite{blastz}. In many
cases the alignment of non coding sequences is considered in the
broader context of aligning two genome sequences, such as those
detailed in the analysis of the mouse genome sequence
\shortcite{Waterston} and much of the novel machinery has been focused
on the pragmatic problems posed mainly by the length of the genome
sequences.


Given such activity, we have been surprised by the apparent
disinterest in the specific underlying model of current alignment
programs. We expect that when two protein structures have a common
ancestor, there are a series of co-linear ``alignable'' residues,
which are presumed to be homologous bases in each extant sequences
with a smooth process of substitution explaining the differences which
maintain the structure or function of the protein sequence. These
alignable residues are separated by unalignable gaps in each
sequence. If they do not share a common ancestor then there are no
homologous base pairs and so no biologically valid alignment. In non
coding alignments many of these assumptions are not true. Firstly
there is ample evidence in non coding regions that either
translocations and inversions both occur naturally and also often do
not perturb the function of the sequence.  This observation means that
resulting alignments of homologous base pairs may not be
colinear. Secondly, rather than the relatively large colinear
stretches of sequence of protein domains being the indivisible unit of
protein sequences, far smaller motifs, being the binding site for
particular DNA or RNA recognising factors are the indivisible units,
which are often well modelled by small stretches of sequence with few
if any gaps. Finally the combination of the small size of motifs and
the functional tolerance of relative position and orientation allows
for de-novo creation and removal of motifs to support a broader
functional region. Effectively this process of de novo creation of
functional regions is never considered in protein alignments, and has
intriguing consequences for the alignment of sequences where
equivalent functional sequences, which are still related by a smooth
process of subsitution and selection across a small region of sequence
are not necessarily a subset of the orthologous base pairs. Evidence
of such birth-death process of motifs have been found in careful
experimentally verified studies \shortcite{Manolis}


Standard alignment programs therefore try find the maximal co-linear
set of aligned residues under some scoring scheme which usually
optimises for a mixture of small to mid-sized gaps. This is likely to
be completely inappropiate for certain types non coding DNA. Such
alignment programs cannot direct tolerate inversions or
translocations, and it is only after post-processing the local
processes (which may be optimised to an entirely inappropiate model)
of finding long co-linear regions that such inversions or
translocations are potentially visible. This prevents any observations
of more interesting, potentially real cases of alignment that break
co-linearity or score poorly under a mid-sized gap model. Previous
studies have explored scoring schemes for non coding DNA
\shortcite{Jareborg,Hardison}. This work reuses the DBA pair-HMM
method presented in \shortcite{Jareborg} which has also been used in a
number of other studies \shortcite{OtherDBAUsingPapers}. In this work
we examine relaxing the need for strict co-linearity of the aligned
regions, designing a new program, \emph{Promoterwise}. We show by
simulation that this program can reliably find inverted and
translocated sequences. We apply this to a number of genome-wide
alignments of promoters and confirm the longstanding view that
transcription factors are under tight transcriptional regulation; in
particular in the vertebrate clade, many transcription factors have
large portions of their promoters conserved between mammalia and
telosts. In contrast other gene classes, including some well
understood promoters do not seem to have such preserved alignable
regions and we propose some explanations for this apparent divergence.


\subsection{Methods}

The alignment of two sequences allow both inversions and translocations
is NP-complete \shortcite{Sankoff}. Therefore we took a greedy approach
of proposing a large number of believable sub alignments and then reconciled
them by accepting the higher scoring alignments and rejecting alignments
with significant overlap. An outline of the method is given in pseudocode
below:

\begin{verbatim}

  foreach stand in sequence-a
    hsp-list = Generate-All-HSPs (sequencea,sequenceb)
      Find all 6 out 7 base pair matches, extend to HSPs
      using a +5,-4 scoring scheme. Reject HSPs with a
      score < H

   sub-alignment-list = align-HSPs (hsp-list) 
     sort HSPs by score
     foreach remaining hsp
      initialise with top scoring HSP

      descend the list of unaccount HSPs, and
      merge if they are consistent with current
      region within expansion parameter E 

      once no more HSPs are mergable, align
      the region with the DBA style pair HMM

  sort all subalignments (both strands) by score
    accept the first alignment
    foreach remaining alignment
      reject if alignment overlaps any already
      accepted alignment in either sequence

\end{verbatim}

Although any potential alignment model can be used instead
of the DBA pair HMM, an important feature of the DBA model is
that it is conservatively parameterised, such that random sequence
is rarely aligned. This ensures that the greedy-style rejection
of alignments on the basis of overlap is unlikely to be due to a small
additional extension of an alignment due to random matches. In contrast,
more standard smith-waterman alignments can have extensive random
extensions, in particular on the edges of real alignments.

\subsection{Results}

Ben's stuff - simulate evolution with allowing translocations and
flips; score smith-waterman, DBA and promoterwise on these cases,
produce sensitivity vs specificity style plots.
 

Laurence's stuff Human/Fugu, Human/Zebra, Zebra/Fugu comparisons, Tables of 
interpro obs vs expected chi2, also GO (with experimental evidence codes) in 
human.

How many times do we see non-colinear arrangements in these cases? Any
time? (probably not!)


\subsection{Discussion}
     
     
Wax lyrical about inversions and translocations

speculate wildly about the need for transcription factors to have
transcription regulation.


\section{Acknowledgements}

L.E, B.P and E.B. are funded by EMBL. We used spare compute resource
from the Ensembl project, funded by the Wellcome Trust.


\bibliography{alignment}



\end{document}
